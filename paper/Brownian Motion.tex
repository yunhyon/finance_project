%% LyX 2.3.3 created this file.  For more info, see http://www.lyx.org/.
%% Do not edit unless you really know what you are doing.
\documentclass[english]{article}
\usepackage[latin9]{inputenc}
\usepackage[letterpaper]{geometry}
\geometry{verbose}
\usepackage{amstext}
\usepackage{amssymb}

\makeatletter
%%%%%%%%%%%%%%%%%%%%%%%%%%%%%% User specified LaTeX commands.
%no date
\date{}

%No Indent
\usepackage{indentfirst}
\setlength{\parindent}{0pt}

%Math packages
\usepackage{fourier} % Use the Adobe Utopia font for the document - comment this line to return to the LaTeX default
\usepackage[english]{babel} % English language/hyphenation
\usepackage{amsmath,amsfonts,amsthm} % Math packages

%Theorem/Definition/Lemma/Corrolary and etc
\newtheorem{Definition}{Definition}
\newtheorem{Lemma}{Lemma}
\newtheorem{Proof}{Proof}
\newtheorem{Corollary}{Corollary}
\newtheorem{Theorem}{Theorem}
\newtheorem{Remark}{Remark}
\newtheorem{Proposition}{Proposition}

\makeatother

\usepackage{babel}
\begin{document}
\title{Brownian Motion}

\maketitle
Brownian motion is a mathematical definition based on probability
theory. Here is the mathematical definition of Brownian motion.

\begin{Definition} \label{Definition of Brownian Motion} Suppose
there is a stochastic process $W(t,\omega)$ defined on a probability
space, $\left(\Omega_{1}^{R},\mathcal{F},\mathbb{P}\right)$ . We
say $W(t,\omega)$ is a Brownian motion if

(i) $W_{0}(\omega)=0$

(ii) For all $0=t_{0}<t_{1}<...<t_{m}=T$ , the increments $W_{t_{1}}(\omega)-W_{t_{0}}(\omega)$,
$W_{t_{2}}(\omega)-W_{t_{1}}(\omega),...,W_{t_{m}}(\omega)-W_{t_{m-1}}(\omega)$
are independent and each of these increments is normally distributed
with $\mathbb{E}[$$W_{t_{i+1}}(\omega)-W_{t_{i-1}}(\omega)]=0$ and
$Var[$$W_{t_{i+1}}(\omega)-W_{t_{i-1}}(\omega)]=t_{i+1}-t_{i}$.

\end{Definition}

\begin{Definition} \label{Definition of Sample Space} Let $\mathcal{A}$
be the set of all continuous functions $f:[0,\infty)\rightarrow\mathbb{R}$.
The n-random-path-sample space, denoted by $\Omega_{n}^{R}$ is defined
as $\Omega_{n}^{R}=\left\{ \left(f_{1},f_{2},...,f_{n}\right)\;|\;f_{i}\in\mathcal{A}\textrm{ for }i=1,2,...,n\right\} $.

\end{Definition}

\begin{Definition} \label{Definition of Sigma Algebra} Let $\mathcal{F}$
be a non-empty collection of subsets of $\Omega_{n}^{R}$. We say
that $\mathcal{F}$ is a $\sigma$-algebra if 

(i) It is closed under complementation. That is, if a set $\mathcal{S}\in\mathcal{F}$,
then its complement $\mathcal{S}^{c}\in\mathcal{F}$.

(ii) It is closed under countable unions. That is, if there is a sequence
of sets in $\mathcal{F}$ , $\mathcal{S}_{1},\mathcal{S}_{2},...\in\mathcal{F}$,
then their countable unions $\cup_{n=1}^{\infty}\mathcal{A}_{n}\in\mathcal{F}$. 

\end{Definition} 

\begin{Definition} \label{Definition of Measurable Space and Probability Space}
If a function $\mathbb{P}:\mathcal{F}\rightarrow[0,1]$ satisfies 

(i) $\mathbb{P}\left(\Omega\right)=1$ and

(ii) If there is a sequence of disjoint sets in $\mathcal{F}$, $\mathcal{S}_{1},\mathcal{S}_{2},...\in\mathcal{F}$,
then $\mathbb{P}\left(\cup_{n=1}^{\infty}\mathcal{S}_{n}\right)=\sum_{n=1}^{\infty}\mathbb{P}\left(\mathcal{S}_{n}\right)$.

then we say $\mathbb{P}$ is a probability measure. We say $\left(\Omega_{n}^{R},\mathcal{F}\right)$
is a measurable space and $\left(\Omega_{n}^{R},\mathcal{F},\mathbb{P}\right)$
is a probability space.

\end{Definition}

\begin{Definition} \label{Definition of Random Variable on Sample Space}
A random variable $X\left(\omega\right)$ defined on $\Omega_{n}^{R}$
is a function $X:\Omega_{n}^{R}\rightarrow\mathbb{R}^{d}$ where $\mathbb{R}^{d}$
denotes a d-dimensional Euclidean space. 

\end{Definition}

\begin{Definition} \label{Definition of measurable Random Variable }
If a $\sigma$-algebra is the smallest $\sigma$-algebra containing
all the subsets of defined as $\{X\in B\}:=\{\omega\in\Omega\;|\;X\left(\omega\right)\in B\textrm{\textrm{ for }every }B\in\mathcal{B}(\mathbb{R}^{d})\}$,
we say the $\sigma$- algebra is generated by $\sigma\left(X(\omega)\right)$.
$\mathcal{B}(\mathbb{R}^{d})$ denotes the smallest $\sigma$-algebra
containing all open sets of $\mathbb{R}^{d}$. If every set in $\sigma\left(X(\omega)\right)$
is in a $\sigma$-algebra $\mathcal{G}$, we say $X(\omega)$ is a
$\mathcal{G}$-measurable random variable.

\end{Definition}

\begin{Definition} \label{Definition of Random Variable on Measurable Space}
A random variable $X(\omega)$ defined on $(\Omega_{n}^{R},\mathcal{F},\mathbb{P})$
is a real-valued function $X:\Omega_{n}^{R}\rightarrow\mathbb{R}^{d}$
with the property that every subset of $\Omega_{n}^{R}$ defined as
$\{X\in B\}$ is in the $\sigma$-algebra $\mathcal{F}$.

\end{Definition}

\begin{Definition} \label{Definition of Stochastic Process} A stochastic
process $X(t,\omega):=\{X_{t}(\omega)\;|\;0\leq t\leq T\}$ defined
on $(\Omega_{n}^{R},\mathcal{F},\mathbb{P})$ is a collection of the
random variables $X_{t}(\omega)$ defined on $(\Omega_{n}^{R},\mathcal{F},\mathbb{P})$
indexed by time $t\in[0,T].$

\end{Definition}


\end{document}
